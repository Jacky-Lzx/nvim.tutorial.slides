\documentclass[aspectratio=169]{ctexbeamer}
\definecolor{urls}{RGB}{137, 180, 250}
\definecolor{link_text}{RGB}{245, 224, 220}
\hypersetup{
  colorlinks,
  linkcolor=, % This config controls the jumps inside the pdf
  urlcolor=urls,
}
\renewcommand{\UrlFont}{\ttfamily\scriptsize}

\usetheme{AnnArbor}
\usepackage[style=Mocha,accent=Rosewater]{beamercolorthemecatppuccin}

\usefonttheme{serif}
\usefonttheme{professionalfonts}

\usepackage[T1]{fontenc}
\setmainfont{LXGW WenKai}
% \setmainfont{Cascadia Code NF}
% \setsansfont{}
\setmonofont{Cascadia Code NF}
\usepackage{xeCJK}
\setCJKmainfont{LXGW WenKai}
% \setCJKmainfont{}
\setCJKmonofont{Cascadia Code NF}
\newcommand{\nerd}[1]{\texttt{#1}}
\setmonofont{Cascadia Code NF}[
  Contextuals=Alternate
]

\PassOptionsToPackage{hyphens}{url}
% \usepackage{ulem}
\usepackage{graphicx}
%\usepackage{wrapfig}
\usepackage{pifont} % Symbols used as itemize symbols
\usepackage{enumitem}
\setlist[itemize,1]{label={\small\color[RGB]{242, 205, 205}\ding{111}}}
\setlist[itemize,2]{label={\footnotesize\color[RGB]{242, 205, 205}\ding{111}}}
\setlist[itemize,3]{label={\scriptsize\color[RGB]{242, 205, 205}\ding{111}}}
\usepackage{float}
\usepackage{booktabs}

\setbeamerfont{footnote}{size=\tiny}
\setbeamertemplate{footnote}{%
  \color[RGB]{108, 112, 134}%
  \insertfootnotetext%
}
\setlength{\footnotesep}{0.3\baselineskip}
\newcommand{\refnote}[1]{\footnotetext{#1}}

\usetheme{AnnArbor}

\usepackage{pifont}

\usepackage{amsmath, amssymb, amsthm}
\usepackage{listings}
\lstdefinestyle{bash}{
  alsoletter=-,
  keywordstyle=[2]{\color[RGB]{243, 139, 168}},
  morekeywords=[2]{sudo},
  keywordstyle=[3]{\color[RGB]{166, 227, 161}},
  morekeywords=[3]{add-apt-repository, apt-get, apt},
  keywordstyle=[4]{\color[RGB]{250, 179, 135}},
  morekeywords=[4]{install},
}
\lstdefinestyle{lua}{
  alsoletter=-,
  keywordstyle=[2]{\color[RGB]{137, 180, 250}},
  morekeywords=[2]{name, priority, opts, config, dependencies, submodules, main, version, init, number, boolean, opts_extend},
  keywordstyle=[3]{\color[RGB]{180, 190, 254}},
  morekeywords=[3]{fun, setup, ensure_installed, highlight, indent, enable},
  keywordstyle=[4]{\color[RGB]{250, 179, 135}},
  morekeywords=[4]{},
}
\lstdefinestyle{path}{
  alsoletter=~,
  basicstyle={\footnotesize\ttfamily\color[RGB]{147, 153, 178}\itshape},
}
\lstset{
  language={[5.1]lua},
  style=lua,
  basicstyle=\footnotesize\ttfamily,
  breaklines=true,
  showstringspaces=false,
  breakatwhitespace=true,
  keywordstyle=\color[RGB]{245, 169, 127},
  numberstyle={\ttfamily\color[RGB]{110, 115, 141}},
  commentstyle={\color[RGB]{147, 153, 178}\itshape},
  stringstyle={\color[RGB]{166, 218, 149}},
}
% NOTE: \lstinline{} command does not support background color
\lstdefinestyle{nvim}{
  alsoletter=:,
  keywordstyle=[3]{\color[RGB]{166, 227, 161}},
  morekeywords=[3]{:Tutor, :help}, % ChkTeX 26
}

\newcommand{\TODO}[1]{\textcolor{red}{TODO\@: #1} }

% \newcommand{\link}[3][]{\href{#3}{#2}\footnote[#1]{\url{#3}}}
\newcommand{\link}[3][]{\href{#3}{#2\textsuperscript{\nerd{}}}}

\title{Neovim从入门到出门(番外篇)}
\subtitle{1. Lua语言配置}
\author{Jacky-Lzx}
\date{\today}

\usepackage{tikz}
\titlegraphic {
  \begin{tikzpicture}[overlay,remember picture]
    \node at (-6, 4.5){
      \includegraphics[height=1cm]{./Figures/Neovim_logo.png}
    };
    \node at (6, 4.5){
      \includegraphics[height=1cm]{./Figures/Catppuccin_logo.png}
    };
  \end{tikzpicture}
}

\usepackage{makecell}

% ChkTeX-file 19
\begin{document}

\begin{frame}
  \titlepage
\end{frame}
\begin{frame}{大纲}
  \tableofcontents
\end{frame}
% Current section
\AtBeginSection[ ] {
  \begin{frame}{大纲}
    \tableofcontents[currentsection]
  \end{frame}
}

\section{Lazy.nvim插件配置(二)}

\begin{frame}{Lazy.nvim插件配置(二)}
  \begin{itemize}
    \item 关于lazy.nvim如何配置插件的内容在第二节中已经介绍
    \item 现存的问题
      \begin{itemize}
        \item 一个语言相关的配置散布在多个文件的多个配置中,难以维护
        \item 例如,Lua语言对应的配置设置有
          \begin{itemize}
            \item mason.nvim的配置
            \item conform.nvim的配置
            \item lazydev.nvim插件
            \item nvim-lspconfig的配置
          \end{itemize}
      \end{itemize}
    \item 解决方法
      \begin{itemize}
        \item 将所有Lua语言相关的配置集中到一个文件中
        \item 使用插件重载的方法,使涉及到的通用插件不受影响
      \end{itemize}
  \end{itemize}
\end{frame}

\begin{frame}{插件重载}
  \begin{itemize}
    \item 推荐阅读:AstroNvim对于自定义插件配置的描述

      \url{https://docs.astronvim.com/configuration/customizing_plugins}
    \item 插件重载
      \begin{itemize}
        \item 当配置文件中存在两个及以上同一个插件的配置表时,lazy.nvim如何确定该插件的配置方式
      \end{itemize}
  \end{itemize}
\end{frame}

\begin{frame}[fragile]{插件重载 - 示例}
  \begin{columns}
    \fontsize{8pt}{10pt}\selectfont

    \begin{column}[t]{0.5\linewidth}
      \begin{itemize}
        \item 假设我们在两个文件中对nvim-treesitter这个插件编写了配置
          \begin{lstlisting}[basicstyle=\tiny\ttfamily]
-- ./lua/plugins/lsp.lua
-- Loaded first
{
  "nvim-treesitter/nvim-treesitter",
  opts = {
    ensure_installed = { "lua", "vim" },
    highlight = { enable = true },
  },
}

-- ./lua/plugins/lang/python.lua
-- Loaded later
{
  "nvim-treesitter/nvim-treesitter",
  opts = {
    ensure_installed = { "python" },
    highlight = { enable = false },
    indent = { enable = false },
  },
}
          \end{lstlisting}
      \end{itemize}
    \end{column}

    \begin{column}[t]{0.5\linewidth}
      \begin{itemize}
        \item 在lazy.nvim中的最终配置
          \begin{lstlisting}[basicstyle=\tiny\ttfamily]
{
  "nvim-treesitter/nvim-treesitter",
  opts = {
    ensure_installed = { "python" },
    highlight = { enable = false },
    indent = { enable = false },
  },
}
          \end{lstlisting}
        \item 错误:\lstinline{ensure_installed}的值被覆盖
        \item 错误原因:lazy.nvim在融合插件配置时,会将table类型的配置项进行合并,将其他类型的配置项进行覆盖
          \begin{itemize}
            \item \lstinline{opts}:进行合并
            \item \lstinline{highlight}:进行合并
            \item \lstinline{indent}:进行合并
            \item \lstinline{ensure_installed}:list类型,进行覆盖
          \end{itemize}
      \end{itemize}
    \end{column}
  \end{columns}
\end{frame}

\begin{frame}[fragile]{插件重载 - 解决方法一}
  \begin{columns}
    \begin{column}{0.5\linewidth}
      \begin{itemize}
        \item 解决方法一
          \begin{lstlisting}[basicstyle=\tiny\ttfamily]
-- ./lua/plugins/lsp.lua
opts = {
  ensure_installed = { "lua", "vim" },
}

-- ./lua/plugins/lang/python.lua
opts = function(_, opts)
  table.insert(opts.ensure_installed, "python")
end
          \end{lstlisting}
        \item 最终效果
          \begin{lstlisting}[basicstyle=\tiny\ttfamily]
opts = {
  ensure_installed = { "lua", "vim", "python" },
}
          \end{lstlisting}
      \end{itemize}
    \end{column}

    \begin{column}{0.5\linewidth}
      \begin{itemize}
        \item 使用\lstinline{opts}函数来修改配置
          \begin{lstlisting}[basicstyle=\tiny\ttfamily]
opts = function(_, opts)
  -- ...
end
          \end{lstlisting}
          \begin{itemize}
            \item 若函数无返回值,则lazy.nvim会将\lstinline{opts}作为最终的\lstinline{opts}配置
            \item 若函数有返回值,则lazy.nvim会将返回值作为最终的\lstinline{opts}配置
          \end{itemize}
      \end{itemize}
    \end{column}
  \end{columns}
\end{frame}

\begin{frame}[fragile]{插件重载 - 解决方法二}
  \begin{columns}
    \begin{column}{0.5\linewidth}
      \begin{itemize}
        \item 解决方法二
          \begin{lstlisting}[basicstyle=\tiny\ttfamily]
-- ./lua/plugins/lsp.lua
opts = {
  ensure_installed = { "lua", "vim" },
},
opts_extend = { "ensure_installed" },

-- ./lua/plugins/lang/python.lua
opts = {
  ensure_installed = { "python" },
}
opts_extend = { "ensure_installed" },
          \end{lstlisting}
        \item 最终效果
          \begin{lstlisting}[basicstyle=\tiny\ttfamily]
opts = {
  ensure_installed = { "lua", "vim", "python" },
}
          \end{lstlisting}
      \end{itemize}
    \end{column}

    \begin{column}{0.5\linewidth}
      \begin{itemize}
        \item 使用\lstinline{opts_extend}来指定需要融合的list项
        \item 指定的list项将进行融合,而不是覆盖
      \end{itemize}
    \end{column}
  \end{columns}
\end{frame}

\section{Lua语言相关的插件配置}

\begin{frame}[fragile]{nvim-treesitter}
  \begin{columns}
    \begin{column}[t]{0.5\linewidth}
      \begin{itemize}
        \item 基础设置
          \begin{lstlisting}[basicstyle=\tiny\ttfamily]
-- ./lua/plugins/treesitter.lua
{
  "nvim-treesitter/nvim-treesitter",
  opts = {
    auto_install = true,
    ensure_installed = { "c", "vim", "vimdoc", "query", "elixir", "heex", "javascript", "html" },
    sync_install = false,
    highlight = { enable = true },
    indent = { enable = true },
  },
  opts_extend = { "ensure_installed" },
},
          \end{lstlisting}
      \end{itemize}
    \end{column}

    \begin{column}[t]{0.5\linewidth}
      \begin{itemize}
        \item Lua语言设置
          \begin{lstlisting}[basicstyle=\tiny\ttfamily]
-- ./lua/plugins/lang/lua.lua
{
  "nvim-treesitter/nvim-treesitter",
  optional = true,
  opts = {
    ensure_installed = { "lua" },
  },
  opts_extend = { "ensure_installed" },
},
          \end{lstlisting}
      \end{itemize}
    \end{column}
  \end{columns}
\end{frame}

\begin{frame}[fragile]{mason.nvim}
  \begin{columns}
    \begin{column}[t]{0.5\linewidth}
      \begin{itemize}
        \item 基础设置
          \begin{lstlisting}[basicstyle=\tiny\ttfamily]
-- ./lua/plugins/lsp.lua
{
  "williamboman/mason.nvim",
  opts = {
    ensure_installed = {},
  },
  opts_extend = { "ensure_installed" },
  config = function(_, opts)
    -- ...
  end,
},
          \end{lstlisting}
      \end{itemize}
    \end{column}

    \begin{column}[t]{0.5\linewidth}
      \begin{itemize}
        \item Lua语言设置
          \begin{lstlisting}[basicstyle=\tiny\ttfamily]
-- ./lua/plugins/lang/lua.lua
{
  "williamboman/mason.nvim",
  optional = true,
  opts = {
    ensure_installed = {
      "lua-language-server",
      "stylua",
    },
  },
  opts_extend = { "ensure_installed" },
},
          \end{lstlisting}
      \end{itemize}
    \end{column}
  \end{columns}
\end{frame}

\begin{frame}[fragile]{conform.nvim}
  \begin{columns}
    \begin{column}[t]{0.5\linewidth}
      \begin{itemize}
        \item 基础设置
          \begin{lstlisting}[basicstyle=\tiny\ttfamily]
-- ./lua/plugins/lsp.lua
{
  "stevearc/conform.nvim",
  event = "BufWritePre",
  opts = {
    formatters_by_ft = {
      ["_"] = { "trim_whitespace" },
    },
    -- ...
  },
  init = function()
    -- ...
  end,
},
          \end{lstlisting}
      \end{itemize}
    \end{column}

    \begin{column}[t]{0.5\linewidth}
      \begin{itemize}
        \item Lua语言设置
          \begin{lstlisting}[basicstyle=\tiny\ttfamily]
-- ./lua/plugins/lang/lua.lua
{
  "stevearc/conform.nvim",
  optional = true,
  opts = {
    formatters_by_ft = {
      lua = { "stylua" },
    },
  },
},
          \end{lstlisting}
      \end{itemize}
    \end{column}
  \end{columns}
\end{frame}

\begin{frame}[fragile]{lazydev.nvim}
  \begin{itemize}
    \item lazydev.nvim从lsp.lua文件移至lua.lua文件中
      \begin{lstlisting}[basicstyle=\tiny\ttfamily]
-- ./lua/plugins/lang/lua.lua
{
  "folke/lazydev.nvim",
  ft = "lua", -- only load on lua files
  opts = {
    library = {
      -- See the configuration section for more details. Load luvit types when the `vim.uv` word is found
      { path = "${3rd}/luv/library", words = { "vim%.uv" } },
    },
  },
},
      \end{lstlisting}
  \end{itemize}
\end{frame}

\begin{frame}
  \begin{itemize}
    \item 感谢:
      \begin{itemize}
        \item \link{Catppuccin}{https://catppuccin.com/} \includegraphics[height=10pt]{./Figures/Catppuccin_logo.png}
        \item \link{Catppuccin for beamer}{https://github.com/atticus-sullivan/beamercolortheme}
      \end{itemize}
      \vspace{0.5cm}
    \item 本教程的全部材料可以在我的Github上找到
      \begin{itemize}
        \item Slides: \url{https://github.com/Jacky-Lzx/nvim.tutorial.slides}
        \item Config: \url{https://github.com/Jacky-Lzx/nvim.tutorial.config}
      \end{itemize}
  \end{itemize}
\end{frame}

\end{document}
